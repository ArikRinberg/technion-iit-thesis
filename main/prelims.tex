\chapter{Preliminaries}
\label{chap:prelims}

A preliminaries chapter is not necessary, but it may be a good idea to use it for presenting your theoretical/mathematical framework in a more detailed and technical way than the introduction, and to perhaps establish some basic lemmata/observations common to multiple chapters of your thesis.

\section{Some section}

Let's define some concept we'll be using throughout the thesis.

\begin{definition}
The \emph{von Neumann model} of a computer, also known as the \emph{Princeton architecture} is an architecture for digital computers, which consists of a processing units, containing an ALU and processing registers; a control unit consisting of an instruction register and a program counter; a memory unit which stores both data and instructions; and input-and-output mechanisms.
\end{definition}

\section{Acronyms and abbreviations}

When defining a \gls{tla} for the first time (or an acronym with any number of
letters), the \verb|\gls{tla}| macro of the \texttt{glossaries-extra} package
generates its full form, with its acronym in parentheses. The acronym is defined
in the \texttt{abbrevs.tex} file. The next time we use \gls{tla}, only its
short form is printed. \Glspl{etla} can be capitlized or use the plural form
with macros such as \verb|\Gls| or \verb|\glspl|.

In addition, the \verb|\gls| macro adds the pages where a term is used to the
\nameref{chap:notation-and-abbreviations} \autoref{chap:notation-and-abbreviations}.