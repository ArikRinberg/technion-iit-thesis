\section{Intermediate value linearizability}
\label{ivl-sec:ivl}

Section~\ref{ivl-ssec:definitions} introduces definitions that we utilize to define IVL.
Section~\ref{ivl-ssec:ivl} defines IVL for deterministic algorithms and proves that it is a local property.
Section~\ref{ivl-ssec:sivl} extends IVL for randomized algorithms, and Section~\ref{ivl-ssec:comparisons}
compares IVL to other correctness criteria.

\subsection{Definitions}
\label{ivl-ssec:definitions}

Our definitions use the notion of skeleton histories:
A \emph{skeleton history} is a sequence of invocation and response events, where the return values
of the responses are undefined, denoted $?$. For a history $H$, we define the operator $H^?$
as altering all response values in $H$ to $?$, resulting in a skeleton
history.
% Consider a randomized algorithm $A$, a schedule $\sigma$, and
% two coin flip vectors $\vv{c}, \vv{c}' \in \Omega^{\infty}$. From the uniform
% step complexity assumption, $\left[H(A, \vv{c}, \sigma)\right]^?=\left[H(A, \vv{c}', \sigma)\right]^?$,
% i.e., the skeleton history of a randomized algorithm's execution is uniquely determined by
% the schedule $\sigma$.

In this paper we discuss correctness criteria for a class of objects we call
\emph{quantitative}. These are objects that support two operations: (1) {\sc update},
which captures all mutating operations \inred{that do} not return a value; and (2) {\sc query},
which returns a value from some \emph{partially ordered set (poset)}. A poset
defines a partial relation between elements from the set, denoted $\leq$.
In Section~\ref{ivl-sec:examples} we exemplify our definitions for three types of return
values: (1) numerical values (a totally ordered domain), (2) sets of elements ordered
by containment (a partially ordered domain), and (3) ranked items stored in a priority
queue (a totally ordered domain).

In a \emph{deterministic quantitative object}
the return values of {\sc query} operations are uniquely defined. Namely, the object's sequential specification $\mathcal{H}$
contains exactly one history for every sequential history skeleton $H$; we denote this
history by $\tau_\mathcal{H}(H)$. Thus, $\tau_\mathcal{H}(H^?) = H$ for every history $H\in \mathcal{H}$.
Furthermore, for every sequential skeleton history $H$, by definition, $\tau_\mathcal{H}(H) \in \mathcal{H}$.
\inred{Note that if the object is not deterministic, $\tau_\mathcal{H}(H^?)$ isn't uniquely defined. We
use the following example to show how skeleton histories can be linearized and then converted back to histories:}

\begin{example}

Consider an execution in which a batched counter (formally defined in Section~\ref{ivl-ssec:lower-bound}) initialized to $0$
is incremented by $3$ by process $p$ concurrently with a query by
process $q$, which returns $0$. Its history is:
\[ H = inv_p(inc(3)), inv_q(query), rsp_p(inc), rsp_q(query \rightarrow 0). \]
The skeleton history $H^?$ is:
\[ H^? = inv_p(inc(3)), inv_q(query), rsp_p(inc), rsp_q(query \rightarrow ?). \]
A possible linearization of $H^?$ is:
\[ H'=inv_p(inc(3)), rsp_p(inc), inv_q(query), rsp_q(query \rightarrow ?). \]
Given the sequential specification $\mathcal{H}$ of a batched counter,
we get:
\[ \tau_\mathcal{H}(H')=inv_p(inc(3)), rsp_p(inc), inv_q(query), rsp_q(query \rightarrow 3). \]
In a different linearization, the query may return $0$ instead.

\end{example}

\subsection{Intermediate value linearizability}
\label{ivl-ssec:ivl}

% We consider objects supporting two operations, {\sc update} and {\sc query},
% where the former captures all mutating operations. Furthermore, all return
% values are from a totally ordered domain. We call such objects \emph{quantitative}
% objects. The return values in each sequential
% execution, i.e., the sequential specification $\mathcal{H}$, can be defined as a sequential
% deterministic algorithm $A$. By slight abuse of notation, we use $A$ as the
% sequential specification in this case. Given a deterministic algorithm
% $A$, this allows us to embody a sequential skeleton history $H^?$ by adding the
% correct return values, denoted $H^?(A)$, i.e., executing the operations on $A$ in the order
% that they happen in $H^?$. By definition, $H^?(A) \in \mathcal{H}$.

% We formalize correctness criteria for objects whose return values are from
% some \emph{partially ordered set (poset)}. A poset defines a partial relation between
% elements from the set, denoted $\leq$. An example of a poset is a power set over $S$,
% i.e., the set of all sets. In this case, the relation is $\subseteq$. A quantitative object is
% an example of such an object, as the \textsc{query}
% operation returns a value from a totally ordered domain.
We now define intermediate value linearizability.

% \begin{definition}[Intermediate value linearizability]
%   A history $H$ of an object is IVL with respect to sequential specification $\mathcal{H}$ if there
%   exist two linearizations $H_1, H_2$ of $H^?$ such that for every operation $o$ that returns a
%   value from some partially ordered set $(S, \leq)$ in $H$,
%   \[\text{ret}(o, \tau_\mathcal{H}(H_1)) \leq \text{ret}(o, H) \leq \text{ret}(o, \tau_\mathcal{H}(H_2)). \]

%   Algorithm $A$ is an \emph{IVL implementation} of a sequential specification $\mathcal{H}$ if every
%   history of a well-formed execution of $A$ is IVL with respect to $\mathcal{H}$.
%   \label{ivl-def:ivl}
% \end{definition}
    

\begin{definition}[Intermediate value linearizability]
  A history $H$ of an object is IVL with respect to sequential specification $\mathcal{H}$ if there
  exist two linearizations $H_1, H_2$ of $H^?$ such that for every {\sc query} $Q$
  that returns in $H$,
  \[\text{ret}(Q, \tau_\mathcal{H}(H_1)) \leq \text{ret}(Q, H) \leq \text{ret}(Q, \tau_\mathcal{H}(H_2)). \]

  Algorithm $A$ is an \emph{IVL implementation} of a sequential specification $\mathcal{H}$ if every
  history of a well-formed execution of $A$ is IVL with respect to $\mathcal{H}$.
  \label{ivl-def:ivl}
\end{definition}

Note that a linearizable object is trivially IVL, as the skeleton history of the linearization of $H$ plays the roles
of both $H_1$ and $H_2$. We emphasize that in a sequential execution, an IVL object is not relaxed
in any way -- it must follow the sequential specification. Similarly, in a well-formed history,
operations of the same process never overlap, and so IVL executions satisfy program order.

\inred{We now show that IVL is both local and non-blocking(as defined in~\cite{herlihy1990linearizability}):}
%The following theorem shows that this property is local (as defined in~\cite{herlihy1990linearizability}):
\begin{restatable}{thm}{local}
  \label{ivl-thm:ivl-local}
  A history $H$ of a well-formed execution of algorithm $A$ over a set of objects $\mathcal{X}$
  is IVL if and only if for each object $x \in \mathcal{X}$, $H|_x$ is IVL.
\end{restatable}
\begin{proof}
  Let $A$ be a deterministic algorithm, and let $\mathcal{H}_x$ be the sequential specification of object $x$, for every $x \in \mathcal{X}$.
  The ``only if'' part is immediate.

  Denote by ${H_1^x}, {H_2^x}$ the linearizations of ${H|_x}^?$ given by the
  definition of IVL.
  We first construct a linearization $H_1$ of $H^?$, \inred{i.e., the order $\prec_{H_1}$},
  as follows: For every pending operation on object $x$, we either
  add the corresponding response or remove it based on ${H_1^x}$. We then construct a partial order
  of operations as the union of $\{{\prec}_{H_1^x}\}_{x \in \mathcal{X}}$ and the realtime order
  of operations in $H$. As each ${\prec}_{H_1^x}$ must adhere to the realtime order of $H|_x$,
  and therefore $H$, and the set of operations $\{{\prec}_{H_1^x}\}_{x \in \mathcal{X}}$ are disjoint, this partial order
  is well defined. Consider two concurrent operations $op_1, op_2$ in $H$. If they
  do not belong to the same history $H|_x$ for some object $x$, we order them arbitrarily in $\prec_{H_1}$.
  We construct linearization $H_2$ of $H^2$ by defining the order of operations $\prec_{H_2}$ in a similar fashion,
  \inred{ where the added responses or removed pending operations are based on ${H_2^x}$ instead of ${H_1^x}$}.

  By construction all invocations and responses appearing in $H^?$ appear both in $H_1$ and in $H_2$,
  and $H_1$ and $H_2$ preserve the partial order $\prec_{H^?}$. Therefore, $H_1$ and $H_2$ are linearizations
  of $H^?$.
  
  Consider some read $R$ on some object $x \in \mathcal{X}$ that returns in $H$. As $H|_x$ is IVL
  $\text{ret}(R,  \tau_{\mathcal{H}_x}(H_1^x)) \leq \text{ret}(R, H|_x) \leq \text{ret}(R, \tau_{\mathcal{H}_x}(H_2^x))$.
  Note that $\text{ret}(R, H|_x) = \text{ret}(R, H)$.
  Furthermore, $\text{ret}(R, \tau_{\mathcal{H}_x}(H_1^x))= \text{ret}(R, \tau_{\mathcal{H}}(H_1))$
  and $\text{ret}(R, \tau_{\mathcal{H}_x}(H_2^x))= \text{ret}(R, \tau_{\mathcal{H}}(H_2))$,
  as objects other than $x$ do not affect the return value of this operation. Therefore $H$ is IVL.
\end{proof}
% \begin{theorem}
%   A history $H$ of a well-formed execution of algorithm $A$ over a set of objects $\mathcal{X}$
%   is IVL if and only if for each object $x \in \mathcal{X}$, $H|_x$ is IVL.
% \end{theorem}
% \begin{proof}
%   Let $A$ be a deterministic algorithm, and let $\mathcal{H}_x$ be the sequential specification of object $x$, for every $x \in \mathcal{X}$.
%   The ``only if'' part is immediate.

%   Denote by ${H_1^x}, {H_2^x}$ the linearizations of ${H|_x}^?$ guaranteed by the
%   definition of IVL.
%   We first construct a linearization $H_1$ of $H^?$, defined
%   by the order $\prec_{H_1}$ as follows: For every pending operation on object $x$, we either
%   add the corresponding response or remove it based on ${H_1^x}$. We then construct a partial order
%   of operations as the union of $\{{\prec}_{H_1^x}\}_{x \in \mathcal{X}}$ and the realtime order
%   of operations in $H$. As ${\prec}_{H_1^x}$ must adhere to the realtime order of $H|_x$,
%   and therefore $H$, and the order of operations in $\{{\prec}_{H_1^x}\}_{x \in \mathcal{X}}$ are disjoint, this partial order
%   is well defined. Consider two concurrent operations $op_1, op_2$ in $H$. If they
%   do not belong to the same history $H|_x$ for some object $x$, we order them arbitrarily in $\prec_{H_1}$.
%   We construct linearization $H_2$ of $H^2$ by defining the order of operations $\prec_2$ in a similar fashion.

%   By construction all invocations and responses appearing in $H^?$ appear both in $H_1$ and in $H_2$,
%   and $H_1$ and $H_2$ preserve the partial order $\prec_{H^?}$. Therefore, $H_1$ and $H_2$ are linearizations
%   of $H^?$.
  
%   Consider some read $R$ on some object $x \in \mathcal{X}$ that returns in $H$. As $H|_x$ is IVL
%   $\text{ret}(R,  \tau_{\mathcal{H}_x}(H_1^x) \leq \text{ret}(R, H|_x) \leq \text{ret}(R, \tau_{\mathcal{H}_x}(H_2^x))$.
%   Note that $\text{ret}(R, H|_x) = \text{ret}(R, H)$. Furthermore, $\text{ret}(R, \tau_{\mathcal{H}_x}(H_i^x))= \text{ret}(R, \tau_{\mathcal{H}}(H_i))$ for $i \in \{1,2\}$,
%   as objects other than $x$ do not affect the return value of this operation. Therefore $H$ is IVL.
% \end{proof}

Locality allows system designers to reason about their system in a modular fashion. Each object can be built separately,
and the system as a whole still satisfies the property.
\inred{
\begin{theorem}
  \label{ivl-thm:ivl-non-blocking}
  Let $H$ be an IVL history of a well-formed execution of a algorithm $A$. If $inv_p(op(arg))$ is an invocation
  of a pending operation in $H$, then there exists $rsp_p(op)\rightarrow ret$ such that $H \cdot rsp_p(op)\rightarrow ret$
  is IVL.
\end{theorem}
\begin{proof}
  Let $H$ be an IVL history of a well-formed execution of a algorithm $A$. Let $inv_p(op(arg))$ be an invocation
  of a pending operation in $H$. If the invocation is not that of a {\sc query} operation, then
  $H \cdot rsp_p(op)\rightarrow \bot$ is IVL. Otherwise,  denote the query as $Q$.
  
  Let $H^?$ be the skeleton history of $H$. Denote $H' = H^? \cdot rsp_p(op)\rightarrow ?$.
  Note that $H'$ is also a skeleton history. Let $H''$ be some linearization of $H'$. Finally, let
  $v=\text{ret}(Q, \tau_\mathcal{H}(H''))$. Then, $H \cdot rsp_p(op)\rightarrow v$ is IVL,
  where $H_1=H_2=H''$.
\end{proof}
}

\subsection{Extending IVL for randomized algorithms}
\label{ivl-ssec:sivl}

% the specification of the deterministic algorithm $A(\vv{c})$ is given
% by the deterministic sequential specification $\mathcal{H}(\vv{c})$.
% Thus, a randomized sequential specification is a distribution over deterministic sequential specifications
% $\mathcal{H}(\vv{c})$ with coin flip vectors $\vv{c} \in \Omega^\infty$, i.e., a distribution over

% A randomized sequential specification is a distribution over deterministic sequential specifications
% $\mathcal{H}(\vv{c})$ with coin flip vectors $\vv{c} \in \Omega^\infty$. Recall that we consider
% a weak adversary in this paper. Under such an adversary,
% a randomized sequential algorithm A satisfies $\mathcal{H}$ if for
% every schedule $\sigma$, under a randomly sampled coin flip
% vector $\vv{c} \in \Omega^\infty$, $H(A,\vv{c},\sigma) \in \mathcal{H}(\vv{c})$. This
% means that for a given schedule $\sigma$, the distribution of return values
% in $A(\vv{c})$ is the same as in $\mathcal{H}(\vv{c})$.

\inred{For the remainder of this paper we consider the strongest progress guarantee, bounded wait-freedom.
\inred{An operation $op$ is \emph{wait-free} if whenever a process $p$ invokes $op$, $op$ returns
a response regardless of steps taken by other processes.}
An operation $op$ is \emph{bounded wait-free} if whenever any process $p$
invokes $op$, $op$ returns a response in a bounded number of
$p$'s steps, regardless of steps taken by other processes.
An operation's \emph{step-complexity} is the maximum number of steps
a process takes during a single execution of this operation. We
can convert every bounded wait-free algorithm to a \emph{uniform step complexity}
one, in which each operation takes the exact same
number of steps in every execution.
This can be achieved by padding shorter
execution paths with empty steps before returning.
Note that in a randomized algorithm with uniform step complexity, coin flips have
no impact on $op$'s execution times.}

In a randomized algorithm $A$ with uniform step complexity, every invocation
of a given operation returns after the same number of steps,
regardless of the coin flip vector $\vv{c}$. This, in turn, implies that
for a given $\sigma$, for any $\vv{c}, \vv{c}' \in \Omega^{\infty}$, the
arising histories $H(A, \vv{c}, \sigma)$ and $H(A, \vv{c}', \sigma)$ differ only in the operations'
return values but not in the order of invocations and responses, as the latter is
determined by $\sigma$, so their skeletons are equal. For randomized algorithm $A$ and schedule $\sigma$,
we denote this arising skeleton history by $H^?(A, \sigma)$.
% Furthermore, for any two coin flip vectors $\vv{c}, \vv{c}' \in \Omega^{\infty}$,
% $H^?(A, \sigma) \triangleq \left[H(A, \vv{c}, \sigma)\right]^?=\left[H(A, \vv{c}', \sigma)\right]^?$.

We are faced with a dilemma when defining the specification
of a randomized algorithm $A$, as the algorithm itself is a distribution over a
set of algorithms $\{A(\vv{c})\}_{\vv{c}\in \Omega^\infty}$. Without knowing
the observed coin flip vector $\vv{c}$, the execution behaves unpredictably. We therefore
define a deterministic sequential specification $\mathcal{H}(\vv{c})$ for each coin flip vector
$\vv{c} \in \Omega^\infty$, so the sequential specification is a
probability distribution on a set of sequential histories $\{\mathcal{H}(\vv{c})\}_{\vv{c}\in \Omega^\infty}$.

% For randomized algorithm $A$, a schedule $\sigma$, we denote this skeleton history
% by $H^?(A, \sigma)$, i.e., $[H(A, \sigma, \vv{c})]^?=H^?(A , \sigma)$
% for any $\vv{c} \in \Omega^\infty$.

% and a coin flip vector $\vv{c} \in \Omega^\infty$

% Therefore, for a randomized algorithm $A$, a schedule $\sigma$, and
% two coin flip vectors $\vv{c}, \vv{c}' \in \Omega^{\infty}$, 
% $H^?(A, \sigma) \triangleq \left[H(A, \vv{c}, \sigma)\right]^?=\left[H(A, \vv{c}', \sigma)\right]^?$.
% We denote this skeleton history by $H^?(A, \sigma)$.

% In contrast to deterministic objects, where we defined a linearization for each history,
% randomized algorithms require a common linearization for each skeleton history
% that will hold true under all possible coin flip vectors.
% We define
% strong linearizability using the linearization of the skeleton history:
% \begin{definition}[Strong linearizability]
%   Consider a skeleton history $H=H^?(A, \sigma)$ of some
%   randomized algorithm $A$ with schedule $\sigma$.
%   $H$ is \emph{strongly linearizable} with respect to $\{\mathcal{H}(\vv{c})\}_{\vv{c} \in \Omega^\infty}$ if there exists a
%   linearization $H'$ of $H$ such that for every coin flip vector $\vv{c}$ and read $R$
%   that returns in $H$: $\text{ret}(R, \tau_{\mathcal{H}(\vv{c})}(H')) = \text{ret}(R, H(A, \vv{c}, \sigma))$.

%   Algorithm $A$ is a \emph{strongly linearizable implementation} of a sequential specification distribution $\{\mathcal{H}(\vv{c})\}_{\vv{c} \in \Omega^\infty}$ if every
%   skeleton history of a well-formed execution of $A$ is strongly linearizable with respect to $\{\mathcal{H}(\vv{c})\}_{\vv{c} \in \Omega^\infty}$.
% \end{definition}
% As linearizable objects are, in particular, IVL, we must strengthen IVL in a similar manner. We call this
% stronger variant \emph{strong intermediate value linearizability (SIVL)}:

A correctness criterion for randomized objects needs to capture the property that the distribution of
a randomized algorithm's outcomes matches the distribution of behaviors allowed by the specification.
Consider, e.g., some sequential skeleton history $H$ of an object defined by $\{\mathcal{H}(\vv{c})\}_{\vv{c}\in \Omega^\infty}$. Let Q be a query
that returns in $H$, and assume that $Q$ has some probability $p$ to return a value $v$ in $\tau_{\mathcal{H}(\vv{c})}(H)$ for
a randomly sampled $\vv{c}$. Intuitively, we would expect that if a randomized algorithm $A$ ``implements'' the specification
$\{\mathcal{H}(\vv{c})\}_{\vv{c}\in \Omega^\infty}$, then $Q$ has a similar probability to return $v$ in sequential executions of $A$ with the same history,
and to some extent also in concurrent executions of $A$ of which $H$ is a linearization. In other words, we
would like the distribution of outcomes of $A$ to match the distribution of outcomes in $\{\mathcal{H}(\vv{c})\}_{\vv{c}\in \Omega^\infty}$.
\inred{For example, consider a sequential specification that allows returns $0$ with probability $0.5$, and $1$ with probability $0.5$. For an
algorithm to implement such an object, it is not enough for it to return only $0$ or $1$, but we expect $0$ to be returned with
probability $0.5$, and $1$ to be returned with probability $0.5$.}


We observe that in order to achieve this, it does not suffice to require that each history has an
arbitrary linearization as we did for deterministic objects, because this might not preserve the
desired distribution \inred{as is shown in~\cite{Wojciech}}. Instead, for randomized objects we require a common linearization for each
skeleton history that will hold true under all possible coin flip vectors.
\inred{In other words, the linearization is independent of the coin flip vector.}
We therefore formally
define IVL for randomized objects as follows:

% \begin{definition}[IVL for randomized algorithms]

%   Consider a skeleton history $H=H^?(A, \sigma)$ of some
%   randomized algorithm $A$ with schedule $\sigma$.
%   $H$ is \emph{IVL} with respect to $\{\mathcal{H}(\vv{c})\}_{\vv{c} \in \Omega^\infty}$ if there exist
%   linearizations $H_1, H_2$ of $H$ such that for every coin flip vector $\vv{c}$ and some operation $o$
%   that returns a value from some partially ordered set $(S, \leq)$ in $H$,
%   \[\text{ret}(o, \tau_{\mathcal{H}(\vv{c})}(H_1)) \leq \text{ret}(o, H(A, \vv{c}, \sigma)) \leq \text{ret}(o, \tau_{\mathcal{H}(\vv{c})}(H_2)). \]

%   Algorithm $A$ is an \emph{IVL implementation} of a sequential specification
%   distribution $\{\mathcal{H}(\vv{c})\}_{\vv{c} \in \Omega^\infty}$ if every skeleton
%   history of a well-formed execution of $A$ is IVL with
%   respect to $\{\mathcal{H}(\vv{c})\}_{\vv{c} \in \Omega^\infty}$.
%   \label{ivl-def:sivl}
% \end{definition}

\begin{definition}[IVL for randomized algorithms]
  Consider a skeleton history $H=H^?(A, \sigma)$ of some
  randomized algorithm $A$ with schedule $\sigma$.
  $H$ is \emph{IVL} with respect to $\{\mathcal{H}(\vv{c})\}_{\vv{c} \in \Omega^\infty}$ if there exist
  linearizations $H_1, H_2$ of $H$ such that for every coin flip vector $\vv{c}$ and query $Q$
  that returns in $H$,
  \[\text{ret}(Q, \tau_{\mathcal{H}(\vv{c})}(H_1)) \leq \text{ret}(Q, H(A, \vv{c}, \sigma)) \leq \text{ret}(Q, \tau_{\mathcal{H}(\vv{c})}(H_2)). \]

  Algorithm $A$ is an \emph{IVL implementation} of a sequential specification
  distribution $\{\mathcal{H}(\vv{c})\}_{\vv{c} \in \Omega^\infty}$ if every skeleton
  history of a well-formed execution of $A$ is IVL with
  respect to $\{\mathcal{H}(\vv{c})\}_{\vv{c} \in \Omega^\infty}$.
  \label{ivl-def:sivl}
\end{definition}

% Note that the query $Q$ in Definition~\ref{ivl-def:sivl} is not necessarily ``correct''; rather,
% it is bounded between two (possibly erroneous) results returned in linearizations.
\inred{Since we require a common linearization of the skeleton histories under all coin flips vectors,
the linearizations are a foriori independent of future coin flips. Hence, an adversary cannot affect
the linearization based on observing the coin flip vector.}

% Since we require a common linearization under all coin flip vectors, we do not need to
% strengthen IVL for randomized settings in the manner that strong linearizability
% strengthens linearizability. This is because the linearizations we consider are a fortiori independent of future coin flips.
% \inred{In other words, the range of admissible behaviors is determined by variation in the interleaving steps, and not
% by the variation  in coin flips. Definition~\ref{ivl-def:sivl} is both local and non-blocking, with a similar proof to Definition~\ref{ivl-def:ivl}.}

% Note that a strongly linearizable object is trivially SIVL, as the linearization of $H$ plays the role of both $H_1$ and $H_2$.

% We show that for a certain family of objects, the error of a sequential
% object remains bounded in a concurrent SIVL implementation.
% We first define the family, which we denote \emph{$\delta$-bounded} randomized
% objects, as follows. Let $o$ be a sequential randomized object, and let $H^?$ be
% some arising skeleton history from schedule $\sigma$.
% Consider any read $R$ that completes in $H^?$. We say that $R$ is bounded if
% there exist $v_1, v_2$ such that with probability
% $\geq 1-\delta$, $R$ returns $v$ in $H^?_{\vv{c}}$ such that $v_1 \leq v \leq v_2$,
% for some sampled $\vv{c} \in \Omega^{\infty}$. If this is true for every completed read $R$ in $H^?$,
% then we say that $H^?$ is $\delta$-bounded. If every history arising for an
% execution of $o$ is $\delta$-bounded, then we say that
% $o$ is a \emph{$\delta$-bounded} randomized object.

% We now show the following theorem:
% \begin{theorem}
%   Let $o$ be a sequential object, and let algorithm $A$ be an SIVL implementation
%   of $o$. If $o$ is a $\delta$-bounded randomized object, then $A$ is a $2\delta$-bounded
%   randomized object.
%   \label{ivl-thm:SIVL-bound}
% \end{theorem}
% \begin{proof}
%   Let $o$ be a $\delta$-bounded randomized sequential object, and let $A$ be an SIVL implementation
%   of $o$. Let $\vv{c}$ be a coin flip vector, and let $\sigma$ be a schedule. Consider $H(\vv{c}, \sigma)$,
%   the concurrent history arising from an execution of $A$. Consider some operation $R$ that returns some value in $H(\vv{c}, \sigma)$.

%   Consider the skeleton history $H^?$ of $H(\vv{c}, \sigma)$. As $A$ is SIVL there exist linearizations
%   $H_1^?, H_2^?$ of $H^?$, such that
%   $\text{ret}(R, {H_1^?}(\vv{c})) \leq \text{ret}(R, {H^?}(\vv{c})) \leq \text{ret}(R, {H_2^?}(\vv{c}))$.
%   Note that $H_i^?(\vv{c})$ is a sequential history of $A$. As such, we can execute the operations on object
%   $o$ in the same order and observing the some coin flip vector $\vv{c}$ and arrive at history $H'$
%   such that $H'=H_i^?(\vv{c})$ for $i \in \{1,2\}$. As $o$ is a $\delta$-bounded randomized object, there exit values
%   $v_i^1, v_i^2$ such that $v_i^1 \leq \text{ret}(H_i^?(\vv{c})) \leq v_i^2$ with probability $\geq 1-\delta$, for $i \in \{1,2\}$.
%   Therefore, $v^1 \leq v_1^1$ and with probability $\leq \delta$: $v^2 \geq v_2^2$.
%   As $A$ is an SIVL implementation of $o$, there exist linearizations $H_1,H_2$,
%   and values $v^1,v^2$ such that $v^1 \leq v \leq v^2$. As $H_1$ and $H_2$ are sequential
%   histories, then there exists $v_1^1,v_2^1,v_1^2,v_2^2$ such that with probability
%   $\geq 1 - \delta$: $v_1^1 \leq v^1 \leq v_2^1$ and with probability
%   $\geq 1 - \delta$: $v_1^2 \leq v^2 \leq v_2^2$. Therefore, with probability $\leq \delta$:
%   $\text{ret}(H_1^?(\vv{c})) \leq v_1^1$ and with probability $\leq \delta$: $\text{ret}(H_2^?(\vv{c})) \geq v_2^2$.

%   Therefore,
%   with probability $\geq 1 - 2 \delta$ then both $\text{ret}(H_1^?(\vv{c})) \geq v_1^1$ and $\text{ret}(H_2^?(\vv{c})) \leq v_2^2$,
%   and therefore with probability $\geq 1 - 2 \delta$ then
%   $v_1^1 \leq \text{ret}(H_1^?(\vv{c})) \leq \text{ret}(H^?(\vv{c})) \leq \text{ret}(H_2^?(\vv{c})) \leq v_2^2$.
%   Meaning, choosing $v_1 = v_1^1$ and $v_2 = v_2^2$ proves the theorem.
% \end{proof}



% The importance we see in defining correctness criteria for randomized objects
% is twofold: the first is the ease of concurrent error relaxation and the
% second is the implementation flexibility it provides.
% \inred{ARIK: I'm trying to work out how to say that the error analysis of
% a strongly linearizable object is the same as an atomic one. I'm not sure how to
% say this.}

% Here too the sequential
% analysis carries over to the concurrent
% setting, with an additional relaxation error, i.e., the error that is added
% due to overtaking a bounded number of operations. Like strong linearizability, the analysis
% is relative to a sequential specification which is only applicable to a deterministic
% execution. While the flexibility is larger than that of strong linearizability,
% it is still bounded by a factor of $r$.

%  This framework does not
% leverage existing literature of the error in the sequential setting, which we
% wish to leverage.

\subsection{\texorpdfstring{$r$}{r}-relaxed IVL}
\label{ivl-ssec:relaxed-cm}

Henzinger et al. define $r$-relaxed semantics~\cite{Henzinger}.
Intuitively, each operation on the relaxed object is assigned a \emph{cost}. An
$r$-relaxation allows operations whose costs do not exceed $r$. Rinberg et al.~\cite{rinberg2019fast}
define the cost of some operation $o$ to be the number of operations that precede it in the
concurrent history, but do not precede it in the sequential one.
\inred{Paraphrasing} the definition of the $r$-relaxed
specification for some sequential specification $\mathcal{H}$ from Rinberg et al.~\cite{rinberg2019fast}:
\begin{definition}[r-relaxation]
    A sequential history $H$ is an \emph{$r$-relaxation} of a sequential history $H'$
    if $H$ is comprised of all the invocations in $H'$ and their responses,
    and each invocation in $H$ is preceded by all but at most $r$ of the invocations that precede the 
    same invocation in $H'$. The \emph{$r$-relaxation} of $\mathcal{H}$ is the set of histories
    that have r-relaxations in $\mathcal{H}$:
    
    $\mathcal{H}^r \triangleq $ $\{H'|\exists H\in$$\mathcal{H}$ s.t. $H$ is an $r$-relaxation of $H'\}$.
    \label{ivl-def:r-relaxtion}
\end{definition}
We can plug Definition~\ref{ivl-def:r-relaxtion} into Definition~\ref{ivl-def:sivl} instead
of the sequential specification there to get a specification for $r$-relaxed IVL objects.
For completeness, we define $r$-relaxed IVL:
\begin{definition}[$r$-Relaxed IVL]
  Consider a skeleton history $H=H^?(A, \sigma)$ of some
  randomized algorithm $A$ with schedule $\sigma$.
  $H$ is $r$-relaxed \emph{IVL} with respect to $\{\mathcal{H}^r(\vv{c})\}_{\vv{c} \in \Omega^\infty}$ if there exist
  linearizations $H_1, H_2$ of $H$ such that for every coin flip vector $\vv{c}$ and query $Q$
  that returns in $H$,
  \[\text{ret}(Q, \tau_{\mathcal{H}^r(\vv{c})}(H_1)) \leq \text{ret}(Q, H(A, \vv{c}, \sigma)) \leq \text{ret}(Q, \tau_{\mathcal{H}^r(\vv{c})}(H_2)). \]

  Algorithm $A$ is an $r$-relaxed \emph{IVL implementation} of a sequential specification
  distribution $\{\mathcal{H}(\vv{c})\}_{\vv{c} \in \Omega^\infty}$ if every skeleton
  history of a well-formed execution of $A$ is IVL with
  respect to $\{\mathcal{H}^r(\vv{c})\}_{\vv{c} \in \Omega^\infty}$.
  \label{ivl-def:relaxed-sivl}
\end{definition}
\inred{As Definition~\ref{ivl-def:relaxed-sivl} is IVL with respect to an r-relaxed specification, it is
both local and non-blocking.} In Section~\ref{ivl-ssec:countMin}, we show that this allows us to create more NUMA-friendly objects.



\subsection{Relationship to other relaxations}
\label{ivl-ssec:comparisons}

In spirit, IVL resembles the \emph{regularity} correctness condition for
single-writer registers~\cite{lamport1986interprocess}, where a query must return
either a value written by a concurrent write or the last value written
by a write that completed before the query began. Stylianopoulos  
et al.~\cite{stylianopoulos2020delegation} adopt a similar condition for data sketches, which they
informally describe as follows: ``a query takes into account all completed
insert operations and possibly a subset of the overlapping ones.''
If the object's estimated quantity (return value) is monotonically increasing
throughout every execution, then IVL essentially captures this condition,
while also allowing intermediate steps of a single update to be observed.
But this is not the case in general. Consider, for example, an
object supporting increment and decrement, and a query that occurs concurrently
with an increment and an ensuing decrement. If the query takes only the decrement
into account (and not the increment), it returns a value that is smaller than all legal return values
that may be returned in linearizations, which violates IVL. Our interval-based
formalization is instrumental to ensuring that a concurrent IVL implementation
preserves the probabilistic error bounds of the respective sequential sketch, which we prove in the next section.

Another example of an object specified in the spirit of IVL is Lamport's monotonic
clock~\cite{lamport1990concurrent}, where a read is required to return a value bounded between the clock's
values at the beginning and end of the read’s interval.

Previous work on set-linearizability~\cite{neiger1994set} and
interval-linearizability~\cite{castaneda2018unifying} has also relaxed linearizability,
allowing a larger set of return values in the presence of overlapping operations. The set of possible return
values, however, must be specified in advance by a given state machine; operations' effects
on one another must be predefined.
In fact, interval-linearizability could be used to define IVL on a per-object basis, by defining
a nondeterministic interval-sequential object in which a read operation can return any value in
the interval defined by the update operations that are concurrent with it. In
contrast, \inred{when considering quantitative objects,} IVL is generic and does not require
additional object-specific definitions; it provides an intuitive quantitative bound
on possible return values.

Henzinger et al.~\cite{Henzinger} define
the quantitative relaxation framework, which
allows executions to differ from the sequential specification up to a bounded cost function.
We use this framework when defining relaxed IVL.
Alistarh et al.
expand upon this and define \emph{distributional linearizability}~\cite{alistarh2018distributionally},
which requires a distribution over the internal
states of the object for its error analysis.
Rinberg et al. consider strongly linearizable $r$-relaxed semantics for randomized objects~\cite{rinberg2019fast}.
IVL differs from these definitions in two points: First, a sequential history of an IVL object
must adhere to the sequential specification, whereas in these
relaxations even a sequential history may diverge from the specification. The second is that these relaxations
are measured with respect to a single linearization. We, instead, bound
the return value between two legal linearizations. The latter is the key to preserving the
error bounds of sequential objects, as we next show.