\section{Lower bound for linearizable batched counter}
\label{ivl-ssec:lower-bound}

The incentive for using an IVL batched counter instead of a linearizable one stems
from a lower bound on the step-complexity of a wait-free linearizable batched counter implementation from SWMR registers.
To show the lower bound we first define the binary snapshot object.
A \emph{snapshot object} has $n$ components written by separate processes, and allows a reader to
capture the shared variable states of all $n$ processes instantaneously. We
consider the \emph{binary snapshot object}, in which each state component may be either $0$ or $1$~\cite{hoepman1993binary}. The object
supports the {\sc update}$_i$($v$) and {\sc scan} operations, where the former sets the state of component $i$
to value $v \in \{0,1\}$ and the latter returns all processes states instantaneously.
It is trivial that the {\sc scan} operation must read all states, therefore its lower bound step complexity
is $\Omega(n)$. Israeli and Shriazi~\cite{israeli1998time} show that the {\sc update} step complexity
of any implementation of a snapshot object from SWMR registers is also $\Omega(n)$. This lower bound
was shown to hold also for multi writer registers~\cite{attiya2006complexity}. While
their proof was originally given for a multi value snapshot object, it holds in the binary case as well~\cite{hoepman1993binary}.

\begin{algorithm}
    \begin{algorithmic}[1]
        % \begin{multicols}{2}

        \State local variable $v_i$ \Comment{Initialized to $0$}
        \State shared batched counter object $\mathit{BC}$ \Comment{Initialized to $0$}
        \Statex
        \Procedure{update$_i$}{$v$}
        \State \algorithmicif\ $v_i = v$\ \algorithmicthen\ \textbf{return} \label{ivl-l:skip}
        \State $v_i \gets v$
        \State \algorithmicif\ $v = 1$\ \algorithmicthen\ $\mathit{BC}$.{\sc update}$_i$($2^i$) \label{ivl-l:set-1}
        \State \algorithmicif\ $v = 0$\ \algorithmicthen\ $\mathit{BC}$.{\sc update}$_i$($2^n - 2^i$) \label{ivl-l:set-0}
        \EndProcedure
        % \columnbreak
        \Procedure{scan}{}
        \State $\mathit{sum} \gets \mathit{BC}$.{\sc read}() \label{ivl-l:read} % $\mod 2^n$
        \State $v[0 \dots n-1] \gets [0 \dots 0]$ \Comment{Initialize an array of $0$'s}
        \For{$i : 0 \leq i \leq n-1$}
        \State \algorithmicif\ bit $i$ is set in $\mathit{sum}$\ \algorithmicthen\ $v[i] \gets 1$ \label{ivl-l:check-set}
        \EndFor
        \State \textbf{return} $v[0 \dots n-1]$
        \EndProcedure
    % \end{multicols}
    \end{algorithmic}
    \caption{Algorithm for process $p_i$, solving binary snapshot with a batched counter object.}
    \label{ivl-alg:bs-with-adder}
\end{algorithm}

To show a lower bound on the {\sc update} operation of wait-free linearizable batched counters,
we show a reduction from a binary snapshot to a batched counter in
Algorithm~\ref{ivl-alg:bs-with-adder}. It uses a local variable $v_i$ and a shared batched counter object.
In a nutshell, the idea is to encode the value of the $i^\text{th}$ component
of the binary snapshot using the $i^\text{th}$ least significant bit of the counter.
When the component changes from $0$ to $1$, {\sc update}$_i$ adds $2^i$, and when it changes from $1$ to $0$,
{\sc update}$_i$ adds $2^n - 2^i$. We now prove the following invariant:
\begin{invariant}
    \inred{For any prefix $H'$ of $H$ of length $t$} of a sequential execution of Algorithm~\ref{ivl-alg:bs-with-adder},
    the sum held by the counter is $c \cdot 2^n + \sum_{i=0}^{n-1}v_i2^i$,
    such that $v_i$ is the parameter passed to the last invocation of {\sc update}$_i$ in $H'$ if such invocation
    exists, and $0$ otherwise, for some integer $c \in \mathbb{N}$.
    \label{ivl-inv:sum}
\end{invariant}
\begin{proof}
    We prove the invariant by induction on the length of $H$, i.e., the number of invocations in $H$,
    denoted $t$. As $H$ is a sequential history, each invocation is followed by a response.
    %\par{\textbf{Base:}}
    The base if for $t=0$, i.e., $H$ is the empty execution. In this case no updates
    have been invoked, therefore $v_i=0$ for all $0 \leq i \leq n-1$. The sum returned by the counter
    is $0$. Choosing $c=0$ satisfies the invariant.
    %\par{\textbf{Induction step:}}
    Our induction hypothesis is that the invariant holds for a history of length $t$.
    We prove that it holds for a history of length $t+1$. The last invocation can be either a {\sc scan}, or an {\sc update}($v$)
    by some process $p_i$. If it is a {\sc scan}, then the counter value doesn't change and the invariant
    holds. Otherwise, it is an {\sc update}($v$). Here, we note two cases. Let $v_i$ be $p_i$'s value
    prior to the {\sc update}($v$) invocation. If $v = v_i$, then the {\sc update} returns without altering the sum
    and the invariant holds. Otherwise, $v \neq v_i$. We analyze two cases, $v=1$ and $v=0$. If $v=1$, then $v_i=0$.
    The sum after the update is $c \cdot 2^n + \sum_{i=0}^{n-1}v_i2^i + 2^i=c \cdot 2^n + \sum_{i=0}^{n-1}v_i'2^i$, where
    $v_j'=v_j$ if $j \neq i$, and $v'_i = 1$, and the invariant holds. If $v=0$, then $v_i=1$.
    The sum after the update is $c \cdot 2^n + \sum_{i=0}^{n-1}v_i2^i + 2^n - 2^i = (c+1) \cdot 2^n + \sum_{i=0}^{n-1}v_i'2^i$,
    where $v_j'=v_j$ if $j \neq i$, and $v'_i = 1$, and the invariant holds.
\end{proof}

Using the invariant, we prove the following lemma:
\begin{lemma}
    For any sequential history $H$, if a {\sc scan} returns $v_i$, and {\sc update}$_i$($v$) is the last update invocation in $H$
    prior to the {\sc scan}, then $v_i = v$. If no such update exists, then $v_i=0$.
    \label{ivl-lmma:scan-correctness}
\end{lemma}
\begin{proof}
    Let $S$ be a {\sc scan} in $H'$. Consider the sum $\textit{sum}$ as read by scan $S$.
    From Invariant~\ref{ivl-inv:sum}, the value held by the counter is $c \cdot 2^n + \sum_{i=0}^{n-1}v_i2^i$.
    There are two cases, either there is an update invocation prior to $S$, or there isn't. If there isn't, then by
    Invariant~\ref{ivl-inv:sum} the corresponding $v_i=0$. The process sees bit $i=0$,
    and will return $0$. Therefore, the lemma holds.

    Otherwise, there is a an update prior to $S$ in $H$. As the sum is equal to $c \cdot 2^n + \sum_{i=0}^{n-1}v_i2^i$,
    by Invariant~\ref{ivl-inv:sum}, bit $i$ is equal to $1$ iff the parameter passed to the last invocation of update was $1$.
    Therefore, the scan returns the parameter of the last update and the lemma holds.
\end{proof}

\begin{lemma}
    Algorithm~\ref{ivl-alg:bs-with-adder} implements a linearizable binary snapshot using a linearizable batched counter.
    \label{ivl-lmma:reduction}
\end{lemma}
\begin{proof}
    Let $H$ be a history of Algorithm~\ref{ivl-alg:bs-with-adder}, and let $H'$ \inred{be the subset of operations in $H$ that
    access the linearizable batched counter (and removed otherwise),}
    where each operation is linearized at its access to the linearizable batched counter \inred{with responses added to
    pending operations}, or
    its response if $v_i = v$ on line~\ref{ivl-l:skip}.
    Applying Lemma~\ref{ivl-lmma:scan-correctness} to $H'$, we get $H' \in \mathcal{H}$ and therefore $H$ is linearizable.
\end{proof}

It follows from the algorithm that if the counter
object is bounded wait-free then the {\sc scan} and {\sc update} operations are bounded wait-free. Therefore, the lower
bound proved by Israeli and Shriazi~\cite{israeli1998time} holds, and the {\sc update} must take $\Omega(n)$
steps. Other than the access to the counter in the {\sc update} operation, it takes
$O(1)$ steps. Therefore, the access to the counter object must take $\Omega(n)$ steps. We have proven the following theorem.
\begin{theorem}
    For any linearizable wait-free implementation of a batched counter object with $n$ processes from SWMR registers, the step-complexity
    of the {\sc update} operation is $\Omega(n)$.
    \label{ivl-thm:lower-bound}
\end{theorem}
