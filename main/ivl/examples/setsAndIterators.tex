\subsection{Non-atomic iterators}
\label{ivl-ssec:sets-and-iterators}

While until now we focused on numerical objects, the idea behind IVL can be
expanded to include other types of objects as well. For example, iterators
in map and set data structures (like skiplists and search trees) typically return a
non-atomic scan of the set of keys -- or items -- in the data structure~\cite{arbel2018harnessing, meir2020oak}.
We show that the semantics of such scan operations are naturally captured using IVL.
Consider a data structure supporting three operations, (1) \textsc{insert}, (2) \textsc{delete},
and (3) \textsc{scan}.
The sequential specification $\mathcal{H}_{MAP}$ is straightforward, a \textsc{scan}
returns all elements that were inserted before it and were not deleted before it.

The concurrent semantics are typically defined as follows~\cite{arbel2018harnessing, meir2020oak}:
\begin{definition}[Non-atomic \textsc{Scan} operation semantics]
Consider a scan operation, returning some set $S$. Let $K$ be the elements that have
been inserted prior to the scan's invocation and have not been deleted before the scan. Let
$I$ be the elements being inserted concurrently to the scan, and let $D$ be the set of elements
that are removed concurrently to the scan. Then $K \setminus D \subseteq S \subseteq K \cup I$.
\label{ivl-def:scan}
\end{definition}

This definition resembles IVL with the partial order of set containment. However, IVL also
adheres to program order. So if a thread adds element $a$, then removes it, then 
adds element $b$, an IVL scan result cannot contain both $a$ and $b$, which is allowed
by Definition~\ref{ivl-def:scan}.

We can, nevertheless, use IVL to capture the correctness semantics of such non-atomic
iterators as specified by Definition~\ref{ivl-def:scan}.
To this end, we add an auxiliary history variable both at the concrete level
(the data structure is augmented to track its removals in an auxiliary history variable holding
tombstones) and at the abstract level (the \textsc{scan} in the augmented sequential
specification returns a set including tombstones).
The algorithm augmented
with the auxiliary variable is an IVL implementation of the augmented sequential specification.  
The \textsc{insert}($a$) operation inserts $a$ into the auxiliary variable.
The \textsc{delete}($a$) operation inserts a tombstone for $a$, denoted $-a$, into the auxiliary variable. The
concrete return value is defined via a function $f$ that returns the
set of elements that are included and do not have tombstones in the
auxiliary variable, e.g., $f(\{a,-a,b\}) = \{b\}$.

Formally, let $\mathcal{V}$ be the set of all possible elements, and denote the possible
tombstones as $\mathcal{V}^-=\{-v| v \in \mathcal{V}\}$. The auxiliary variable
is some $S \in 2^{\mathcal{V} \cup \mathcal{V}^-}$. We define $f : 2^{\mathcal{V} \cup \mathcal{V}^-} \mapsto 2^{\mathcal{V}}$  as:
$f(S) = \{v| v \in S \wedge -v \notin S\}$.

Given the sequential specification $\mathcal{H}_{MAP}$ defined above, denote $\mathcal{H}_{T-MAP}$ as
the augmented object. The sequential specification of a \textsc{scan} operation is
also straightforward: a \textsc{scan} returns a set in $2^{\mathcal{V} \cup \mathcal{V}^-}$
consisting of all elements that were
inserted before it, and tombstones for all elements deleted before it.

We prove that IVL semantics with the auxiliary variable is equivalent to Definition~\ref{ivl-def:scan}:
\begin{lemma}   
Consider a history $H$ of a concurrent object implementing $\mathcal{H}_{MAP}$ and \inred{some} scan of it returning a set
of items $S \in 2^{\mathcal{V}}$. Denote by $\mathcal{H}_{T-MAP}$ the object augmented with
the auxiliary history variable tracking tombstones.
\inred{Then $\exists S' \in 2^{\mathcal{V} \cup \mathcal{V}^-}$ such that
returning $H$ is IVL with respect to $\mathcal{H}_{T-MAP}$ and $f(S') = S$.}
\end{lemma}
\begin{proof}
\inred{Let $H$ be a history of some concurrent object implementing $\mathcal{H}_{MAP}$},
and let $S$ be the result
of some non-atomic scan operation in $H$. Let $K$,$I$, and $D$ be as defined by
Definition~\ref{ivl-def:scan}.

$S$ satisfies Definition~\ref{ivl-def:scan}\inred{, as the object implements $\mathcal{H}_{MAP}$}. We show that $\exists S'$ such that $S'$
is IVL with respect to $\mathcal{H}_{T-MAP}$ and $f(S')=S$.

Let $H_1$ be a linearization of $H^?$ where the scan is linearized before all concurrent operations,
and let $H_2$ be a linearization of $H^?$ where the scan is linearized after all concurrent operations.
Let ${S'}_i \in 2^{\mathcal{V} \cup \mathcal{V}^-} $ be the return value of the scan in $\tau_\mathcal{H}(H_i)$,
for $i=1,2$. Let $K'$ be the set of all insertions and deletions before the scan in $H_1$,
then, by construction, $S_1' = K'$ and $S_2' = K' \cup {I \cup D^-}$. Note that $f(K')=K$.

We construct $S'$ as $S_1' \cup I' \cup D'^-$ as follows: $I' = S \cap I$, and $D' = D \setminus S$.
I.e., $I'$ is all inserted elements that are observed by $S$, and $D'$ are all deleted
elements not observed by $S$ -- meaning elements that are deleted concurrently to the scan
and not observed by it. By construction
\[ S_1' \subseteq S' = S_1' \cup I' \cup {D'}^- \subseteq S_1' \cup I \cup D^- = S_2', \]
so returning $S'$ is IVL with respect to $\mathcal{H}_{T-MAP}$.

Furthermore, by construction, $S'=K' \cup (S \cap I) \cup (D \setminus S)^-$, so
$f(S')=K \cup (S \cap I) \setminus (D \setminus S)$. By Definition~\ref{ivl-def:scan}
$K \setminus D \subseteq S \subseteq K \cup I$, we get that
$K \setminus (D \setminus S) \subseteq S \subseteq K \cup (I \cap S)$, and $f(S')=S$, as needed.
\end{proof}

\inred{We believe that instead of giving a long definition of the SCAN operation as is currently
done~\cite{arbel2018harnessing, meir2020oak}, IVL captures the semantics of such operations accurately
and succinctly.}