In this section we define an algorithm property under which the ``hogwild'' parallelism
(i.e., allowing all threads to work concurrently with no synchronization) produces as IVL
algorithm. We note that a quantitative object is essentially a function on its memory
state.

We denote the underlying objects as $\mathcal{X} \triangleq x_1,x_2,\dots, x_k$ for some $k$. An \textsc{update} operation
updates some of subset of objects in $\mathcal{X}$, and \textsc{query} applies some function $f$ and returns
$f(\mathcal{X}) \triangleq f(x_1,x_2,\dots,x_k)$. We define $x_i(t)$ as the state of the object $i$ at time $t$,
and denote $\mathcal{X}(t)$ as $x_1(t), x_2(t), \dots, x_k(t)$.
Finally, we can define a \emph{monotonically increasing} algorithm:
\begin{definition}
    A quantitative algorithm $A$ is \emph{monotonically increasing} if:
    \begin{itemize}
        \item For every index $1 \leq i \leq k$: $x_i(t) \leq x_i(t')$ for all times $t \leq t'$ or $x_i(t) \geq x_i(t')$for all times $t \leq t'$; and
        \item $f(\mathcal{X}(t)) \leq f(\mathcal{X}(t'))$ for all times $t \leq t'$.
    \end{itemize}
\end{definition}

We can similarly define a \emph{monotonically decreasing} algorithm. We can now define a \emph{monotonic}
algorithm as one which is either monotonically increasing or monotonically decreasing.

An important distinction to be made is that we refer to times and not to histories of some length. This is so
we capture intermediate states as well. To illustrate the importance of this distinction, consider an increment
object. Trivially, it is an object in which the histories are monotonically increasing. However, it is not necessarily
a monotonic object under our definition. Consider an implementation consisting of a single register, starting at
$0$ and increasing by one every \textsc{update}. The \textsc{update} implementation can consist of two subsequent
operations: (1) write $-1$ to the register, and (2) write $x+1$ where $x$ is the value before the \textsc{update}. As
there is a time between (1) and (2), the implementation isn't monotonic.

We can now formalize the following theorem:
\begin{theorem}
    Consider some algorithm $A$, and consider its straightforward parallelism $A_c$. If $A$ is monotonic,
    then $A_c$ is IVL.
\end{theorem}

An important subfamily of monotonic algorithms are linear sketches. Linear sketches involve sampling a
matrix $A$ from some distribution $\pi$ over linear maps of size $n \times r$. A vector $x$ of size $n$ is
presented to the algorithm, which maintains a sketch $Ax$. This provides a concise representation of $x$ which
can be queried for various approximate answers. In an online insertion-only stream, an update $(i, \delta_t)$
causes the change $x_i \gets x_i + \delta_t$, and $\delta_t \geq 0$. If, furthermore, all elements in $A$ are
positive, the resulting sketching algorithm is monotonic.