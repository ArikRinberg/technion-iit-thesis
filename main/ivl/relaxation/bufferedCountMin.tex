\subsection{\texorpdfstring{$r$}{r}-Bounded IVL-Relaxed IVL}
\label{ivl-ssec:relaxed-cm}

Using the $r$-relaxed IVL form, we can create a buffered CM sketch
by using a similar framework to Rinberg et al.~\cite{rinberg2019fast}. The
\textsc{update} operations are buffered locally until a certain threshold
is reached. The resulting matrix is then propagated to the shared CM sketch,
which is updated element by element. 

This buffered approach allows for far better use of local memory. Rather than every
\textsc{update} operating on shared memory, most operations are on local memory. This
leads to better cache utilization, and forgoes expensive NUMA memory accesses.

Another difference is the \textsc{query} operation. In Rinberg et al. a \textsc{query}
takes a strongly linearizable snapshot of the current global state, and applies the query
to it.  In a CM sketch, a linearizable snapshot is an expensive operation -- an adversarial
scheduler could theoretically block the operation from ever returning. Using an IVL query
forgoes the need for a strongly linearizable snapshot, while retaining error bounds.

Finally, defining the correctness of such an object requires defining $r$-relaxed IVL.
Borrowing the definition of the $r$-relaxed specification from some sequential specification
$\mathcal{H}$ from~\cite{rinberg2019fast}:
\begin{definition}[r-relaxation]
    A sequential history $H$ is an \emph{$r$-relaxation} of a sequential history $H'$,
    if $H$ is comprised of all but at most $r$ of the invocations in $H'$ and their responses,
    and each invocation in $H$ is preceded by all but at most $r$ of the invocations that precede the 
    same invocation in $H'$. The \emph{$r$-relaxation} of $\mathcal{H}$ is the set of histories
    that have r-relaxations in $\mathcal{H}$:
    
    $\mathcal{H}^r \triangleq $ $\{H'|\exists H\in$$\mathcal{H}$ s.t. $H$ is an $r$-relaxation of $H'\}$.
    \label{ivl-def:r-relaxtion}
\end{definition}
We can plug Definition~\ref{ivl-def:r-relaxtion} into Definition~\ref{ivl-def:sivl} instead of the sequential specification
there to get a specification for $r$-relaxed IVL objects. Let $\mathcal{H}_{CM}$ be the sequential specification of
the CM sketch. We can now say the the buffered IVL CM sketch is $r$-relaxed IVL with respect to $\mathcal{H}_{CM}$, where
$r=2Nb$, where $N$ is the number of \inred{processes} and $b$ is the local buffer size (i.e., the number of updates processed
between propagations).

The error analysis follows from the relaxation and the definition of IVL. Consider some query $Q$ on $a$
that returns $v$, and let $S^{start}$ and $S^{end}$ be the state of the system at the start and end of
$Q$'s execution, respectively. Let $N$ be the length of the stream and $v^{start}$ be the number of times $a$ appears in the stream before $S^{start}$.
\[ v^{start} - r \leq v \leq  v^{end} + \epsilon N^{end}. \]