\section{Conclusion}
\label{ivl-sec:conclusion}

We have presented IVL, a new correctness criterion that provides flexibility
in the return values of quantitative objects while bounding the error that this may
introduce. IVL has a number of desirable properties: First,
like linearizability, it is a local property, allowing designers to reason about each part
of the system separately. Second, also like linearizability but unlike other relaxations of
it, IVL preserves the error bounds of PAC objects. Third, IVL is generically defined for
all quantitative objects, and does not necessitate object-specific definitions. Finally,
IVL is inherently amenable to cheaper implementations than linearizability in some cases.

Via the example of a CountMin sketch, we have illustrated that IVL provides
a generic way to efficiently parallelize data sketches while leveraging their
sequential error analysis to bound the error in the concurrent implementation.

We have shown that IVL can also capture the semantics of non-atomic snapshots,
by augmenting the data structure with an auxiliary history
variable. Finally, we have shown that sometimes IVL is useful
in tandem with other correctness criteria via the
example of a priority queue, where we pair IVL with sequential consistency.

The notion of IVL raises a main question for future research:
In this work we have shown that IVL is a sufficient condition for parallel
$(\epsilon,\delta)$-bounded objects, in that it preserves their sequential
error. It would be interesting to investigate whether IVL is also necessary,
or whether some weaker condition is sufficient.