\subsection{IVL and SC}
\label{ivl-sec:other-objects}

While we focused on quantitative objects, the idea behind IVL can be
expanded to include other types of objects as well. For example, iterators
in \emph{key-value stores (KVS)} typically return a non-atomic scan \inred{CITE OAK (and others)}.
The semantics of such scan operations can be defined using IVL. Consider a KVS supporting
three operations, (1) \textsc{insert}, (2) \textsc{delete}, and (3) \textsc{scan}.
The sequential specification $\mathcal{H}_{KVS}$ is straightforward, a \textsc{scan}
returns all key-value pairs that were inserted before it, that have not since been deleted.

The concurrent semantics would therefore be as follows: a \textsc{scan} returns at least
all key-value pairs that were inserted before it, that are not concurrently being deleted.
In other words, if $KV$ is the set of pairs that we inserted before some scan that returns $S$, and $D$ is the
set of keys that are being concurrently removed, then $KV \setminus D \subseteq S$.

We would also like to capture the semantics of other types of objects that return values from
partially ordered sets. For example, the \textsc{pop} of a priority queue \inred{CITE} returns
element-priority pairs. Consider a concurrent priority queue with $3$ threads, $t_0,t_1,t_2$.
$t_0$ inserts the pair $(e_1,5)$, followed by $t_1$ inserting $(e_2,3)$, followed by $t_0$
inserting $(e_3,4)$. Concurrently to the inserts, $t_2$ issues a \textsc{pop}. Under linearizability
the pop can return either $(e_1,5)$ or $(e_2,3)$. However, from the perspective of $t_0$ there
is a preference for $(e_3,4)$. 

In fact, we can generalize to all operations that return a value from partially ordered sets:
\begin{definition}[Poset IVL (P-IVL)]
    A history $H$ of an object is IVL with respect to sequential specification $\mathcal{H}$ if there
    exist two linearizations $H_1, H_2$ of $H^?$ such that for every operation $o$ that returns a
    value from some partially ordered set $(S, \leq)$ in $H$,
    \[\text{ret}(o, \tau_\mathcal{H}(H_1)) \leq \text{ret}(o, H) \leq \text{ret}(o, \tau_\mathcal{H}(H_2)). \]
    \label{ivl-def:poset-ivl}
\end{definition}

We observe that sequential consistency \inred{CITE?} and P-IVL are incomparable,
as depicted in Figure \inred{XXX}. Sometimes we require both for the return
values to ``make sense''. For example, a pure P-IVL priority queue requires
that a \textsc{pop} return an element who's priority is bounded between
two linearizable ones, but says nothing about the element being one added
to the queue via a \textsc{push}. E.g., if one linearization would return
an element $(e_1,2)$ and another linearization would return an element $(e_2,5)$, then
according to P-IVL, also returning $(e_3,3)$ is legal. But $(e_3,3)$ may be an element
which was never added to the queue.

For the proper semantics we would require that a priority queue be both
P-IVL and sequentially consistent. The former bounds the priority of the returned
elements, and the latter requires the elements to be ones added to the queue.
These criteria compliment one another.