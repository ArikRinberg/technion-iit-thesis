\section{\texorpdfstring{$(\epsilon,\delta)$}{(epsilon,delta)}-bounded objects}
\label{ivl-sec:bounded-objects}

In this section we show that for a large class of randomized objects,
IVL concurrent implementations preserve the error bounds of the respective
sequential ones. More specifically, we focus on randomized
objects like data sketches, which estimate some quantity (or quantities)
with probabilistic guarantees.
Sketches generally support two operations:
{\sc update}($a$), which processes element $a$, and {\sc query}($arg$),
which returns the quantity estimated by the sketch as a function of the previously
processed elements. Sequential sketch algorithms typically have probabilistic error bounds. For example,
the Quantiles sketch estimates the rank of a given element in
a stream within $\pm \epsilon n$ of the true rank, with probability
at least $1-\delta$~\cite{agarwal2013mergeable}, \inred{where $n$ is the number of completed {\sc update} operations
that precede the query.}

We consider in this section a general class of
\emph{$(\epsilon, \delta)$-bounded objects} capturing PAC algorithms. A bounded object's
behavior is defined relative to a deterministic
sequential specification $\mathcal{I}$, which uniquely defines
the \emph{ideal} return value for every query in a sequential
execution. In an $(\epsilon, \delta)$-bounded $\mathcal{I}$ object,
each query returns the ideal return value within
an error of at most $\epsilon$ with probability at least $1-\delta$.
More specifically, it over-estimates (and similarly under-estimates) the
ideal quantity by at most $\epsilon$ with probability at least $1-\frac{\delta}{2}$. 
% Meaning if
% the correct value of a query is $v$, then the $(\epsilon, \delta)$-bounded
% implementation returns a value $\hat{v}$ such that $\hat{v} \geq v-\epsilon$ with probability
% at least $1-\delta$ and $\hat{v} \leq v+\epsilon$ with probability
% at least $1-\delta$. The probability is over the distribution
% of coin flip vectors observed by specific executions. Note that every
% sequential skeleton history can be embodied by a sequential history
% by adding the correct return values with respect to $\mathcal{H}$,
% regardless of the observed coin flip vector.
Formally:
\begin{definition}
    A sequential randomized algorithm $A$ implements an $(\epsilon,\delta)$-bounded $\mathcal{I}$
    object if for every query $Q$ returning in
    an execution of $A$ with any schedule $\sigma$ and a randomly sampled
    coin flip vector $\vv{c} \in \Omega^\infty$,
    \[ ret(Q,H(A,\sigma,\vv{c})) \geq ret(Q, \tau_{\mathcal{I}}(H^?(A,\sigma)) - \epsilon \text{ with probability at least } 1-\frac{\delta}{2},\]
    and 
    \[ ret(Q,H(A,\sigma,\vv{c})) \leq ret(Q, \tau_{\mathcal{I}}(H^?(A,\sigma)) + \epsilon \text{ with probability at least } 1-\frac{\delta}{2}.\]
    \label{ivl-def:seq-e,d-obj}

    $A$ induces a sequential specification $\{A(\vv{c})\}_{\vv{c} \in \Omega^\infty}$ of an $(\epsilon,\delta)$-bounded $\mathcal{I}$ object,
    \inred{by defining the return values of all operations in a given history under a give coin flip vector.}
\end{definition}


We next discuss concurrent implementations of this specification.


To this end, we must specify a correctness criterion on the object's
concurrent executions. As previously stated, the standard notion
is (strong) linearizability, stipulating that we can ``collapse'' each operation in the concurrent
schedule to a single point in time. Intuitively, this means that every query returns a value
that could have been returned by the randomized algorithm at some point during its execution 
interval. So the query returns an $(\epsilon, \delta)$ approximation of the ideal value at that particular point.
But this point is arbitrarily chosen, meaning that the query may return an $\epsilon$ approximation
of any value that the ideal object takes during the query's execution. We therefore look at the minimum and
maximum values that the ideal object may take during a query's interval, and bound the error relative to these values. 


We first define these minimum and maximum values as follows: For a history $H$,
denote by $\mathcal{L}(H^?)$ the set of linearizations of $H^?$.
For a query $Q$ that returns in $H$ and an ideal specification $\mathcal{I}$, we define: 
$$ v^{\mathcal{I}}_{min}(H,Q) \triangleq \min \{ \text{ret}(Q,\tau_\mathcal{I}(L) \mid L \in \mathcal{L}(H^?)\}) ; \\
v^{\mathcal{I}}_{max}(H,Q) \triangleq \max \{ \text{ret}(Q,\tau_\mathcal{I}(L) \mid L \in \mathcal{L}(H^?))\}.
$$
Note that even if $H$ is infinite and has infinitely many linearizations,
because $Q$ returns in $H$, it appears in each linearization by
the end of its execution interval, and therefore $Q$ can return a finite number of different
values in these linearizations, and so the minimum and maximum are well-defined.
Correctness of concurrent $(\epsilon, \delta)$-bounded objects is then formally defined as follows:

\begin{definition}
    A concurrent randomized algorithm $A$ implements an $(\epsilon,\delta)$-bounded
    $\mathcal{I}$ object if for every query $Q$ returning in
    an execution of $A$ with any schedule $\sigma$ and a randomly sampled
    coin flip vector $\vv{c} \in \Omega^\infty$,
    $$ ret(Q,H(A,\sigma,\vv{c})) \geq  v^{\mathcal{I}}_{min}(H(A,\sigma,\vv{c}),Q) - \epsilon \text{ with probability at least } 1-\frac{\delta}{2}, $$
    and 
    $$ ret(Q,H(A,\sigma,\vv{c})) \leq v^{\mathcal{I}}_{max}(H(A,\sigma,\vv{c}),Q) + \epsilon \text{ with probability at least } 1-\frac{\delta}{2}.$$
    \label{ivl-def:con-e,d-obj}
\end{definition} 
    
For simplicity, the definition above uses a common $\epsilon$ for all queries.
While in some algorithms, $\epsilon$ depends on the stream size, i.e.,
the number of updates preceding a query, to avoid cumbersome notations we use
a single variable $\epsilon$, which should be set to the maximum
value that the sketch's $\epsilon$ bound takes during the query's execution interval.
Since the query returns, its execution interval is necessarily bounded, and so
$\epsilon$ is bounded.

The following theorem shows
that IVL implementations allow us to leverage the
``legacy'' analysis of a sequential object's error bounds.
\begin{theorem}
    If $A$ implements an an $(\epsilon,\delta)$-bounded $\mathcal{I}$ object (Definition~\ref{ivl-def:seq-e,d-obj}),
    and $A'$ is an IVL implementation of $A$ (Definition~\ref{ivl-def:sivl}), then $A'$ implements
    a concurrent $(\epsilon,\delta)$-bounded $\mathcal{I}$ object (Definition~\ref{ivl-def:con-e,d-obj}).
    Consider a sequential specification $\{A(\vv{c})\}_{\vv{c} \in \Omega^\infty}$ of an $(\epsilon,\delta)$-bounded
    $\mathcal{I}$ object (Definition~\ref{ivl-def:seq-e,d-obj}). Let $A'$ be an IVL implementation of $A$ (Definition~\ref{ivl-def:sivl}). Then $A'$ implements a concurrent
    $(\epsilon,\delta)$-bounded $\mathcal{I}$ object (Definition~\ref{ivl-def:con-e,d-obj}).
    \label{ivl-thm:SIVL-bound}
\end{theorem}
\begin{proof}
    % Let $A$ and $A'$ be as defined in the theorem.
    Consider a skeleton history $H=H^?(A', \sigma)$ of $A'$ with some schedule $\sigma$, and a
    query $Q$ that returns in $H$. As $A'$ is an IVL implementation of $A$, there exist linearizations
    $H_1$ and $H_2$ of $H$, such that for every $\vv{c}\in\Omega^\infty$,
    $\text{ret}(Q, \tau_{A(\vv{c})}(H_1)) \leq \text{ret}(Q, H(A, \sigma, \vv{c})) \leq \text{ret}(Q, \tau_{A(\vv{c})}(H_2))$.
    %As $\{A(\vv{c})\}_{\vv{c} \in \Omega^\infty}$ captures a sequential $(\epsilon,\delta)$-bounded $\mathcal{I}$ object,
    As $A$ implements a sequential $(\epsilon,\delta)$-bounded $\mathcal{I}$ object,
    % $\text{ret}(Q, \tau_{A(\vv{c})}(H_i)$ is bounded as follows:
    % \[ \text{ret}(Q, \tau_{A(\vv{c})}(H_1)) \geq v_{min}(H(A', \sigma, \vv{c}),Q) - \epsilon \text{ with probability at least } 1-\delta, \]
    % and
    % \[ \text{ret}(Q, \tau_{A(\vv{c})}(H_2)) \leq v_{max}(H(A', \sigma, \vv{c}),Q) + \epsilon \text{ with probability at least } 1-\delta. \]
    $\text{ret}(Q, \tau_{A(\vv{c})}(H_i)$ is bounded as follows:
    \[ \text{ret}(Q, \tau_{A(\vv{c})}(H_1)) \geq \text{ret}(Q, \tau_{\mathcal{I}}(H_1)) - \epsilon \text{ with probability at least } 1-\frac{\delta}{2}, \]
    and
    \[ \text{ret}(Q, \tau_{A(\vv{c})}(H_2)) \leq \text{ret}(Q, \tau_{\mathcal{I}}(H_2)) + \epsilon \text{ with probability at least } 1-\frac{\delta}{2}. \]
    Furthermore, by definition of $v_{min}$ and $v_{max}$:
    \[ \text{ret}(Q, \tau_{\mathcal{I}}(H_1)) \geq  v^{\mathcal{I}}_{min}(H(A', \sigma, \vv{c}),Q) ; \\ 
    \text{ret}(Q, \tau_{A(\vv{c})}(H_2)) \leq v^{\mathcal{I}}_{max}(H(A', \sigma, \vv{c}),Q).\]

    Therefore, with probability at least $1-\frac{\delta}{2}$, $\text{ret}(Q, H(A', \sigma, \vv{c})) \geq v^{\mathcal{I}}_{min}(H(A', \sigma, \vv{c}),Q) - \epsilon$ and
    with probability at least $1-\frac{\delta}{2}$, $\text{ret}(Q, H(A', \sigma, \vv{c})) \leq v^{\mathcal{I}}_{max}(H(A', \sigma, \vv{c}),Q) + \epsilon$, as needed.
    % Therefore $A'$
    % implements a concurrent $(\epsilon,\delta)$-bounded $\mathcal{I}$ object.

    % Let $\{\mathcal{H}(\vv{c})\}_{\vv{c} \in \Omega^\infty}$ be a sequential $(\epsilon,\delta)$-bounded objects specification,
    % and let $\mathcal{H}$ be the ideal specification. Let $A$ be a strongly linearizable implementation
    % of $\{\mathcal{H}(\vv{c})\}_{\vv{c} \in \Omega^\infty}$.
    % Let $\sigma$ be some schedule, consider the concurrent skeleton history $H^?$ arising from
    % the execution of $A$ under $\sigma$. Consider some query $Q$ that returns in $H^?$.

    % As $A$ is SIVL, there exist linearizations $H_1$ and $H_2$ of $H^?$, such that for every
    % $\vv{c}\in\Omega^\infty$ $\text{ret}(Q, \tau_{\mathcal{H}(\vv{c})}(H_1)) \leq \text{ret}(Q, H(A, \sigma, \vv{c})) \leq \text{ret}(Q, \tau_{\mathcal{H}(\vv{c})}(H_1))$.
    % As $\{\mathcal{H}(\vv{c})\}_{\vv{c} \in \Omega^\infty}$ is a sequential $(\epsilon,\delta)$-bounded objects specification,
    % $\text{ret}(Q, \tau_{\mathcal{H}(\vv{c})}(H_i)$ is bounded as per the definition. By definition
    % $\text{ret}(Q, \tau_{\mathcal{H}(\vv{c})}(H_1)) \geq v_{min}(H(A, \sigma, \vv{c}),Q)$ and $\text{ret}(Q, \tau_{\mathcal{H}(\vv{c})}(H_2)) \leq v_{min}(H(A, \sigma, \vv{c}),Q)$.

    % Therefore with probability at least $1-\delta$, $\text{ret}(Q, H(A, \sigma, \vv{c})) \geq v_{min}(H(A, \sigma, \vv{c}),Q) - \epsilon$ and
    % with probability at least $1-\delta$, $\text{ret}(Q, H(A, \sigma, \vv{c})) \geq v_{max}(H(A, \sigma, \vv{c}),Q) + \epsilon$. Therefore $A$
    % implements a concurrent $(\epsilon,\delta)$-bounded object. 
\end{proof}

While easy to prove, Theorem~\ref{ivl-thm:SIVL-bound} shows that IVL is in some sense the ``natural''
correctness property for $(\epsilon,\delta)$-bounded objects; it shows that IVL is a sufficient
criterion for keeping the error bounded. It is less restrictive -- and as we
show below, sometimes cheaper to implement -- than linearizability, and yet strong enough
to preserve the salient properties of sequential executions of $(\epsilon,\delta)$-bounded objects. As
noted in Section~\ref{ivl-ssec:comparisons}, previously suggested relaxations do not inherently guarantee that
error bounds are preserved. For example, regular-like semantics, where a query ``sees''
some subset of the concurrent updates~\cite{stylianopoulos2020delegation},  satisfy IVL (and hence bound
the error) for monotonic objects albeit not for general ones. Indeed, if object values
can both increase and decrease, the results returned under such regular-like semantics can arbitrarily diverge from possible sequential ones.

The importance of Theorem~\ref{ivl-thm:SIVL-bound} is that it allows us to leverage
the vast literature on sequential $(\epsilon, \delta)$-bounded
objects~\cite{morris1978counting, flajolet1985approximate, cichon2011approximate, liu2016one, CountMin, agarwal2013mergeable}
in concurrent implementations. We illustrate this in Section~\ref{ivl-ssec:relaxed-cm} below via the
example of an IVL parallelization of a popular data sketch. By Theorem~\ref{ivl-thm:SIVL-bound},
it preserves the original sketch's error bounds.

% There exists much literature analyzing sequential $(\epsilon, \delta)$-bounded
% objects~\cite{morris1978counting, flajolet1985approximate, cichon2011approximate, liu2016one, CountMin, agarwal2013mergeable}.
% We would like to leverage this literature to analyze the error of the concurrent objects.
% The concurrent implementation correctness criteria defines how ``easy'' it will be to use
% existing sequential analysis. Implementing a strongly linearizable objects means the
% error analysis carries over to the concurrent setting. However, such implementations
% may be non-trivial~\cite{rinberg2019fast,stylianopoulos2020delegation}; using
% a strongly linearizable snapshot may be costly  and, as shown in~\cite{ovens2019strongly},
% these can be costly, both in space and time bounds. Theorem~\ref{ivl-thm:SIVL-bound} proves
% that we are able to leverage existing analyses on $(\epsilon,\delta)$-bounded object,
% such as sketches, while still providing flexibility in implementation, as shown
% in Section~\ref{ivl-sec:countMin}.