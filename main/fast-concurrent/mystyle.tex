% \newtheorem{observation}{Observation}
% \newtheorem{claim}{Claim}
\newtheorem{notation}{Notation}
\newtheorem{invariant}{Invariant}
% \newtheorem{definition}{Definition}
\newtheorem{assertion}{Assertion}
% \newtheorem{thm}{Theorem}
% \newtheorem{lemma}{Lemma}

\newcommand{\inred}[1]{{\color{red}{#1}}}
\newcommand{\inblue}[1]{{\color{blue}{#1}}}
\newcommand{\ingray}[1]{{\color{gray}{#1}}}
\newcommand{\remove}[1]{}
\newcommand\Set[2]{\{\,#1 \mid #2\,\}}
\newcommand\Collection[1]{\{\,#1\,\}}
\newcommand{\abs}[1]{\lvert#1\rvert}
\newcommand{\floor}[1]{{\left\lfloor {#1} \right\rfloor}}
\newcommand{\Int}{\int\limits}
\DeclareMathOperator*{\argmax}{arg\,max}

\declaretheorem[name=Theorem]{rthm}
% \declaretheorem[name=Claim]{clm}

\algblock{Vars}{EndFor}
\algrenewtext{Vars}{variables}
\algrenewtext{EndVars}{}
\algblock{ForEach}{EndFor}
\algrenewtext{ForEach}{for each }

% \newtheorem{observation}{Observation}
%\newtheorem{claim}{Claim}
%\newtheorem{definition}{Definition}
%\newtheorem{lemma}{Lemma}
\newtheorem{crly}{Corollary}


\declaretheorem[name=Claim]{clm}
\declaretheorem[name=Corollary]{crly2}


\newcommand{\CM}[4]{
    \begin{tikzpicture}[baseline=3ex]
        \draw[step=0.5cm,color=gray] (0,0) grid (1,1);
        \node at (0.25,0.25) {#3};
        \node at (0.25,0.75) {#1};
        \node at (0.75,0.75) {#2};
        \node at (0.75,0.25) {#4};
    \end{tikzpicture}
}

\newcommand{\orif}[1]{\textcolor{green}{[#1 -- ORI-FIXED]}}
\newcommand{\ori}[1]{\textcolor{red}{[#1 -- ORI]}}
\newcommand{\arik}[1]{\textcolor{red}{[#1 -- ARIK]}}
\newcommand{\dor}[1]{\textcolor{purple}{[#1 -- DOR]}}
\newcommand{\dorf}[1]{\textcolor{orange}{[#1 -- DOR-FIXED]}}

\newcommand{\forRevThree}[1]{{#1}}


\def \myScale {0.82}

\newcommand{\ingreen}[1]{{#1}}

\newcommand\diag[4]{%
  \multicolumn{1}{p{#2}|}{\hskip-\tabcolsep
  $\vcenter{\begin{tikzpicture}[baseline=0,anchor=south west,inner sep=#1]
  \path[use as bounding box] (0,0) rectangle (#2+2\tabcolsep,\baselineskip);
  \node[minimum width={#2+2\tabcolsep-\pgflinewidth},
        minimum  height=\baselineskip+\extrarowheight-\pgflinewidth] (box) {};
  \draw[line cap=round] (box.north west) -- (box.south east);
  \node[anchor=south west] at (box.south west) {#3};
  \node[anchor=north east] at (box.north east) {#4};
 \end{tikzpicture}}$\hskip-\tabcolsep}}



\newcommand{\fixed}[1]{\vskip 4mm \noindent{\color{blue} \textbf{Response:} {#1}} \vskip 5mm}
%\newcommand{\notfixed}[1]{\todo[inline,linecolor=red,backgroundcolor=red!5,bordercolor=red]{ {\footnotesize NOT FIXED: {#1}}}}

%\newcommand \fixed[1]{{\noindent\color{blue}#1}}

\newcommand \reviewer[1]{{\noindent\color{black}#1}}


%\renewcommand \ori[1]{}
%\renewcommand \janos[1]{}

\newcommand{\hash}{\mathpzc{h}}
%\newcommand{\deth}{\mathpzc{k}}
\newcommand{\deth}{\hat{h}}

\newcommand{\citetext}[1]{\protect\begin{mdframed}[leftmargin=1cm,rightmargin=1cm,shadow=true,shadowcolor=black!20,shadowsize=4pt] {\footnotesize \color{black} #1 } \protect\end{mdframed}}


\newcommand \prob{\mathbb{P}}
%%%%%%%%%%%%%%%%%%%%%%%%%%




\pgfplotsset{compat=1.16}