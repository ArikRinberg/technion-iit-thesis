% This file contains the abstract part of your thesis - in English and
% in Hebrew (within \abstractEnglish and \abstractHebrew respectively).
%
% Notes:
% - This file uses the UTF-8 character set encoding for the Hebrew
%   text not to get garbled. Keep it that way.
% - Assuming your thesis is mainly in English, Graduate School 
%   regulations mandate the following lengths for the abstracts:
%
%      Language    Min. Length   Max. Length
%     ---------------------------------------
%      English       200 words     500 words
%      Hebrew        500 words   2,000 words
%
%   so that the Hebrew abstract typically has some content from
%   the English introduction and an overview of the results, not
%   present in the English; it is not just a translation.

\abstractEnglish{

Processing high-speed, high-volume data often requires performing analytics on a representation of the data, rather than the data itself. To this end, data sketching algorithms have become an indispensable tool for high-speed computations over massive datasets. They maintain the streams' state and answer queries on it (e.g., how many unique elements are in the stream, what is the frequency of elements) using limited memory, at the cost of giving approximations rather than exact answers. Existing libraries provide high optimized sketches, but do not allow parallelism for creating sketches using multiple threads, or querying them while they are being built. In this thesis, we study three aspects of these systems.

We first present a generic approach to parallelizing data sketches efficiently and allowing them to be queried in real time, while bounding the error that such parallelism introduces. Utilizing relaxed semantics and the notion of strong linearizability we prove our algorithm's correctness and analyze the error it induces in some specific sketches. Our implementation achieves high scalability while keeping the error small. We have contributed one of our concurrent sketches to the open-source data sketches library.

Next, we propose a new correctness criterion. The de facto correctness criterion for concurrent objects is linearizability. Intuitively, under linearizability, when a read overlaps an update, it must return the object's value either before the update or after it. Consider, for example, a single batched increment operation that counts three new events, bumping a batched counter's value from $7$ to $10$. In a linearizable implementation of the counter, a read overlapping this update must return either $7$ or $10$. We observe, however, that in typical use cases, any \emph{intermediate} value between $7$ and $10$ would also be acceptable. To capture this additional degree of freedom, we propose \emph{Intermediate Value Linearizability (IVL)}, a new correctness criterion that relaxes linearizability to allow returning intermediate values, for instance $8$ in the example above. Roughly speaking, IVL allows reads to return any value that is bounded between two return values that are legal under linearizability. We show that IVL implementations of sketches inherit the error of their sequential counterpart. To illustrate the power of this result, we give a straightforward and efficient concurrent implementation of a CountMin sketch, which is IVL (albeit not linearizable).

Finally, we study data sketches in the distributed setting. Network measurements are important for identifying congestion, DDoS attacks, and more. To support real-time analytics, stream ingestion is performed jointly by multiple nodes, each observing part of the traffic, periodically reporting its measurements to a single centralized server that aggregates them. To avoid communication congestion, each node reports a compressed version of its collected measurements. Traditionally, nodes symmetrically report summaries of the same size computed on their data. We explain that to maximize the accuracy of the joint measurement, nodes should imply various compression ratios on their measurements based on the amount of traffic observed by each node. We illustrate the approach for three common sketches: The Count-Min sketch (CM), which estimates flow sizes as well as for the K-minimum-values (KMV) sketch and the HyperLogLog (HLL), which both estimate the number of distinct flows. For each sketch, we compute node compression ratios based on the traffic distribution. In general, this is done with a single round of communication with the central server, after which the compression ratio for each node can be computed.
We perform extensive simulations for the sketches and analytically show that, under real-world scenarios, our sketches send smaller summaries than traditional ones while retaining similar error bounds.

} % end of English abstract


\abstractHebrew{

% Note that certain commands don't work that well in Hebrew "mode".
% If this happens to you, try wrapping the command within a
% \textenglish{ } - that may (or may not) help.

כאן יבוא תקציר מורחב בעברית (כאשר שפת החיבור העיקרית היא אנגלית). היקף התקציר יהיה \textenglish{1000-2000} מילים. התקציר יהווה שלמות בפני עצמו ויהיה מובן לקורא בעל ידיעות כלליות בנושא.

בית הספר ללימודי מוסמכים מנחה מספר הנחיות לגבי התקציר בעברית:
\begin{itemize}
\item על התקציר להיכתב במשפטים מקושרים שלמים.
\item בדרך-כלל אין לציין בתקציר מקורות ספרותיים וציטוטים.
\item אין להתייחס למספר של פרק, סעיף, נוסחה, ציור או טבלה שבגוף החיבור, ואין להשתמש בקיצורים, סמלים ומונחים לא מקובלים, אלא אם יש בתקציר די מקום לזיהויים.
\end{itemize}

לעתים יש בכל-זאת יש צורך לכלול פקודה הכוללת קישור פנימי או חיצוני בתוך התקציר העברי; במצבים כאלו כדאי דרך-כלל לעטוף את הפקודה היוצרת את הקישור בתוך פקודת \textenglish{\texttt{\textbackslash{}textenglish\{\}}} כדי למנוע כל מיני פורענויות בלתי-רצויות, כגון כישלון בהידור קובץ ה-\textenglish{PDF} או שימוש בגופן העברי באופן אשר עלול שלא להנעים לעין. לדוגמה: נניח שיש לנו צורך לצטט מקור ביבליוגרפי. אם נעשה זאת סתם-כך: \textenglish{\texttt{\textbackslash{}cite\{Hoeffding\}}}, נקבל: \cite{Hoeffding}; אם נעטוף את פקודת הציטוט, כך: \textenglish{\texttt{\textbackslash{}textenglish\{\textbackslash{}cite\{Hoeffding\}\}}}, נקבל \textenglish{\cite{Hoeffding}} (כפי שהציטוטים נראים גם בטקסט באנגלית).

\subsection*{\texthebrew{תת-חלק בתקציר המורחב}}

תוכן מקוצר לגבי נושא מסוים. התייחסות ל\emph{מושג} מסוים שהחיבור בוחן. וכולי וכולי.


\subsection*{\texthebrew{נקודה מעניינת לגבי העמודים בעברית}}

שימו לב כי העמודים בעברית אמורים להיות מיוצרים בסדר ה''הפוך'', הווה אומר העמוד האחרון בקובץ ה-\textenglish{PDF} הוא הכריכה העברית, לפניו השער העברי, ודפי התקציר צריכים להופיע בסדר הפוך (וכן במספור רומי, לפי נהלי הטכניון). כך אם נתבונן במספר שבתחתית עמוד זה \textenglish{---} אשר צריך להיות העמוד הראשון בתקציר-המורחב מבחינת רצף התוכן, והינו העמוד האחרון מבין עמודי התקציר-המורחב אחרון בקובץ ה-\textenglish{PDF} \textenglish{---} נמצא את המספר \textenglish{i} ...

\newpage

... ואילו עמוד זה של התקציר-המורחב בעברית \textenglish{---} שהינו העמוד השני בתקציר-המורחב מבחינת רצף התוכן, ונמצא ראשון בקובץ ה-\textenglish{PDF} \textenglish{---} ממוספר ב-\textenglish{ii}. המטרה במספור בסדר ה"הפוך" היא, שבעת ההדפסה לא יהיה צורך להפוך דפים, לשנות את סדרם וכולי \textenglish{---} רק להדפיס ולכרוך.

} % end of Hebrew abstract
