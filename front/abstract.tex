
% Just write down your abstract here, no special commands necessary except for the \abstractEnglish{
% before this text is used and a closing } at the end of it

\abstractEnglish{

At this point you write the abstract of your work, in the main language in which it is written (in this template - English). Graduate school regulations require the abstract to constitute an independent whole and be understood to a reader with general knowledge of the field. Use complete sentences and make few or no citations. Do not refer to the main body of the work and do not use uncommon shorthand, symbols and terms unless you have room for explaining them. The English abstract should be between 200 and 500 words long.

So this should contain a few more paragraphs...

\lipsum[10-12]

}

% Note that various commands don't work that well in Hebrew. Specifically,
% if you're using hyperref, you'll have trouble with \url, \autoref, \cite
% and friends. iitthesis-extra.sty has a workaround: the \disabledlinksL
% command. See the .sty for details, or:
% http://tex.stackexchange.com/q/32466/5640 for

\abstractHebrew{

כאן יבוא תקציר מורחב בעברית (כאשר שפת החיבור העיקרית היא אנגלית). היקף התקציר יהיה \textenglish{1000-2000} מילים. התקציר יהווה שלמות בפני עצמו ויהיה מובן לקורא בעל ידיעות כלליות בנושא.

בית הספר ללימודי מוסמכים מנחה מספר הנחיות לגבי התקציר בעברית:
\begin{itemize}
\item על התקציר להיכתב במשפטים מקושרים שלמים.
\item בדרך-כלל אין לציין בתקציר מקורות ספרותיים וציטוטים.
\item אין להתייחס למספר של פרק, סעיף, נוסחה, ציור או טבלה שבגוף החיבור, ואין להשתמש בקיצורים, סמלים ומונחים לא מקובלים, אלא אם יש בתקציר די מקום לזיהויים.
\end{itemize}

לעתים יש בכל-זאת יש צורך לכלול פקודה הכוללת קישור פנימי או חיצוני בתוך התקציר העברי; במצבים כאלו כדאי דרך-כלל לעטוף את הפקודה היוצרת את הקישור בתוך פקודת \textenglish{\texttt{\textbackslash{}textenglish\{\}}} כדי למנוע כל מיני פורענויות בלתי-רצויות, כגון כישלון בהידור קובץ ה-\textenglish{PDF} או שימוש בגופן העברי באופן אשר עלול שלא להנעים לעין. לדוגמה: נניח שיש לנו צורך לצטט מקור ביבליוגרפי. אם נעשה זאת סתם-כך: \textenglish{\texttt{\textbackslash{}cite\{Hoeffding\}}}, נקבל: \cite{Hoeffding}; אם נעטוף את פקודת הציטוט, כך: \textenglish{\texttt{\textbackslash{}textenglish\{\textbackslash{}cite\{Hoeffding\}\}}}, נקבל \textenglish{\cite{Hoeffding}} (כפי שהציטוטים נראים גם בטקסט באנגלית).

\subsection*{\texthebrew{תת-חלק בתקציר המורחב}}

תוכן מקוצר לגבי נושא מסוים. התייחסות ל\emph{מושג} מסוים שהחיבור בוחן. וכולי וכולי.


\subsection*{\texthebrew{נקודה מעניינת לגבי העמודים בעברית}}

שימו לב כי העמודים בעברית אמורים להיות מיוצרים בסדר ה''הפוך'', הווה אומר העמוד האחרון בקובץ ה-\textenglish{PDF} הוא הכריכה העברית, לפניו השער העברי, ודפי התקציר צריכים להופיע בסדר הפוך (וכן במספור רומי, לפי נהלי הטכניון). כך אם נתבונן במספר שבתחתית עמוד זה \textenglish{---} אשר צריך להיות העמוד הראשון בתקציר-המורחב מבחינת רצף התוכן, והינו העמוד האחרון מבין עמודי התקציר-המורחב אחרון בקובץ ה-\textenglish{PDF} \textenglish{---} נמצא את המספר \textenglish{i} ...

\newpage

... ואילו עמוד זה של התקציר-המורחב בעברית \textenglish{---} שהינו העמוד השני בתקציר-המורחב מבחינת רצף התוכן, ונמצא ראשון בקובץ ה-\textenglish{PDF} \textenglish{---} ממוספר ב-\textenglish{ii}. המטרה במספור בסדר ה"הפוך" היא, שבעת ההדפסה לא יהיה צורך להפוך דפים, לשנות את סדרם וכולי \textenglish{---} רק להדפיס ולכרוך.

}
