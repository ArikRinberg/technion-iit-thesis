% Use this file to create "glossary entries" for abbreviations and acronyms.
% The entries defined here don't necessarily have to be used in the thesis.

% For this file to compile (and the example text in the main/prelims.tex file),
% the package glossaries-extra is required. It is automatically included unless
% the noabbrevs class option is used.

% Optionally alter the style for typesetting abbreviations when using the \gls
% command. Note you can also use multiple styles by categorizing abbreviations;
% see the documentation for the glossaries-extras package at:
% https://ctan.org/pkg/glossaries-extra
%
%\setabbreviationstyle[acronym]{long-short-sc}

\newacronym[%
  description=``The Senate and People of Rome'']% The description does not appear anywhere by default
  {spqr}% the key of the acronym (used with the \gls command for example)
  {SPQR}% the short form of the acronym
  {Senātus Populusque Rōmānus}% the long form of the acronym

\newacronym[description=A technology used in data storage devices]%
  {smart}{SMART}{Self-Monitoring, Analysis and Reporting Technology}

\newacronym[description=to build or produce something rather than purchasing it ready-made by others{,} or paying someone else to make it]%
  {DIY}{DIY}{do-it-yourself}

\newacronym[%
  description=a four-letter acronym]
  {etla}
  {ETLA}
  {extended three-letter acronym}

\newabbreviation[%
  description=]% This abbreviation has no description; only the abbreviation and the unabbreviated form will be shown
  {aut}{Aut}{Automorphism group}

\newglossaryentry{symb:c}{%
  name=$c$,%
  description=the speed of light%
}

\newglossaryentry{symb:a-b-closed}{%
  name=\ensuremath{a \pm b},%
  description=the closed interval \ensuremath{\left[a-b,a+b\right]}%
}

\newglossaryentry{supercali}{%
  name=supercalifragilisticexpialidocious,
  description=%
    Atoning for being educable through delicate beauty;
    Something to say when you have nothing to say.}

% --------------------------------

% Commands below will control the behavior/appearance of the list of abbreviations and acronyms

% Uncomment this command to have _all_ abbreviations and acronyms defined
% in this file appear in the final list - rather than just the ones you
% use in the thesis
%\keepUnusedAbbreviations
