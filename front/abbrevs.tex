% Use this file to create "glossary entries" for abbreviations and acronyms.
% The entries defined here don't necessarily have to be used in the thesis.

% For this file to compile (and the example text in the main/prelims.tex file),
% the package glossaries-extra is required. It is automatically included unless
% the noabbrevs class option is used.

% Print long form of acronym first with short in parenthesis
\setabbreviationstyle[acronym]{long-short}

\newacronym[%
    description=``The Senate and People of Rome'']%
    {spqr}{SPQR}{Senātus Populusque Rōmānus}

\newacronym[%
    description=``A technology used in data storage devices'']%
    {smart}{SMART}{Self-Monitoring, Analysis and Reporting Technology}

\newabbreviation[%
  description=]
  {tla}% the key of the acronym (used in \gls macro for example)
  {TLA}% the short form of the acronym
  {three-letter acronym}% the long form of the acronym

\newacronym[%
  description=a four-letter acronym]
  {etla}% the key of the acronym (used in \gls macro for example)
  {ETLA}% the short form of the acronym
  {extended three-letter acronym}% the long form of the acronym

\newglossaryentry{symb:c}{%
  name=$c$,%
  description=the speed of light%
}

\newglossaryentry{symb:a-b}{%
  name=\ensuremath{a \pm b},%
  description=the closed interval \ensuremath{\left[a-b,a+b\right]}%
}

\newglossaryentry{supercali}{%
  name=supercalifragilisticexpialidocious,
  description=%
    Atoning for being educable through delicate beauty.
    Something to say when you have nothing to say.}

% --------------------------------

% Commands below will control the behavior/appearance of the list of abbreviations and acronyms

% Uncomment this command to have _all_ abbreviations and acronyms defined
% in this file appear in the final list - rather than just the ones you
% use in the thesis
%\keepUnusedAbbreviations
